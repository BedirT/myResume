 \documentclass[margin,line]{res}

\usepackage{hyperref}
\hypersetup{
   colorlinks,
   citecolor=black,
   filecolor=black,
   linkcolor=black,
   urlcolor=red
}

\usepackage{url}

\oddsidemargin -.5in
\evensidemargin -.5in
\textwidth=6.0in
\itemsep=0in
\parsep=0in

%if using pdflatex:
\setlength{\pdfpagewidth}{\paperwidth}
\setlength{\pdfpageheight}{\paperheight} 

\newenvironment{list1}{
  \begin{list}{\ding{113}}{%
      \setlength{\itemsep}{0in}
      \setlength{\parsep}{0in} \setlength{\parskip}{0in}
      \setlength{\topsep}{0in} \setlength{\partopsep}{0in} 
      \setlength{\leftmargin}{0.17in}}}{\end{list}}
\newenvironment{list2}{
  \begin{list}{$\bullet$}{%
      \setlength{\itemsep}{0in}
      \setlength{\parsep}{0in} \setlength{\parskip}{0in}
      \setlength{\topsep}{0in} \setlength{\partopsep}{0in} 
      \setlength{\leftmargin}{0.2in}}}{\end{list}}


\begin{document}

\name{M Bedir Tapkan \vspace*{.1in}}

\begin{resume}
\section{\sc Contact Information}
\vspace{.05in}
\begin{tabular}{@{}p{3in}p{3in}}
330 Clareview Station Dr NW & \hfill {\it GitHub:}  \href{https://github.com/bedirT}{bedirT} \\
T5Y 0E6, Edmonton, AB, Canada & \hfill {\it E-mail:}  \href{mailto:tapkan@ualberta.ca}{tapkan@ualberta.ca}\\
& \hfill {\it Portfolio:}  \href{www.bedirtapkan.com}{www.bedirtapkan.com} \\
\end{tabular}


\section{\sc Research Interests}
Multi-connected adaptive agents to solve various complex problems using Reinforcement Learning. Using Deep Reinforcement Learning to close the gap between human learning vs. machine learning.

\section{\sc Education}

% Uncomment for PhD
% {\bf University of British Columbia (UBC)}, Vancouver, BC Canada\\
% {\em Bioinformatics} 
% \begin{list1}
% \item[] {\bf PhD, Bioinformatics, 2019 - 2023}
% \begin{list2}
% \vspace*{.05in}
% \item {\bf Advisor:}  Philipp Lange
% \item {\bf Dissertation Topic:}  ``Rapid and transparent classification of pediatric tumor biopsies by using artificial intelligence guided protein mass spectrometry''

 {\bf University of Alberta (UofA)}, Edmonton, AB Canada\\
 {\em Computer Science} 
 \begin{list1}
 \item[] MsC, Computer Science, 2019 - \textit{Ongoing}
\end{list1}
% \begin{list2}
% \vspace*{.05in}
% \item {\bf Advisor:}  TBD
% \item {\bf Dissertation Topic:}  ``Rapid and transparent classification of pediatric tumor biopsies by using artificial intelligence guided protein mass spectrometry''
%\end{list2}

{\bf North American University (NAU)}, Houston, Texas USA\\
{\em Software Engineering}  \hfill {\bf Overall-GPA: 3.79 - Major-GPA: 3.94}\\
\vspace{-.3cm}
\begin{list1}
\item[] B.S., Computer Science,  2015 - 2018
\end{list1}


\section{\sc Research Experience}

{\bf Undergraduate Research Assistant}

\vspace{-.3cm}
{\em University of Houston} \hfill {\bf January, 2017 - February, 2019}\\ 
\vspace{-.3cm}
\begin{list2}
\item \textbf{Supervisor:} Dr. Ricardo Vilalta
\item Creating a solution that can transfer the previous knowledge to solve new questions, optimization on explore/exploit dilemma for recurring queries. 
\item Using meta-learning to help reinforcement learning models to achieve adaptivity through environment. 
\item Using multi-layered deep reinforcement learning systems that are optimized using meta-learning. 
\end{list2}

\section{\sc Teaching Experience}
{\bf Teaching Assistant}

\vspace{-.3cm}
{\em University of Alberta} \hfill {\bf August 2019 - December 2019}\\
\begin{list2}
	\item TA for CMPUT 355: Algorithms and Puzzles under the supervision of Dr. Ryan Hayward. Helped marking and proctoring, held office hours for students to direct further questions about class material and contributed to the code base development for the class repository.
	\item \textbf{Tools:} Python, Git, Github
\end{list2}
\vspace*{.05in}

{\bf Teacher}

\vspace{-.3cm}
{\em Momentum Learning} \hfill {\bf January 2019 - July 2019}\\
\begin{list2}
\item Taught Python, Java and 3D classes to kids at age 9-15. Focused on fundamentals of programming, how to solve daily problems with certain tools in Computer Science.
\item \textbf{Tools:} Java, Python, TinkerCAD, Git, Github
\end{list2}
\vspace*{.05in}

{\bf Instructor - Algorithms for ICPC}

\vspace{-.3cm}
{\em North American University} \hfill {\bf January 2018 - May 2018}\\
\begin{list2}
\item Taught basic to advanced algorithms, data structures, competitive programming basics, basic logic, math and coding to 12 students from freshman to senior using C++/Python, Git/Github. Prepared class curriculum, lesson plan, and homework assignments.
\item \textbf{Tools:} C++, Python, Git, Github
\end{list2}
\vspace*{.05in}

{\bf Teaching Assistant}

\vspace{-.3cm}
{\em Momentum Learning} \hfill {\bf June 2017 - July 2017}\\
\begin{list2}
\item Co-taught basics of programming to 16 kids from years of age 10 to 18 using Java and Object Oriented programming.
\item \textbf{Tools:} Java, Eclipse
\end{list2}
\vspace*{.05in}

% \section{\sc Publications}
% Kocabas, F., {\bf Ergin, E.K}. (2016). Identification of small molecule binding pocket for inhibition of Crimean-Congo hemorrhagic fever virus OTU protease. {\em Turkish Journal of Biology}, 40:239-249.

%\section{\sc Papers in preparation}
%
% Kocabas, F., Ergin, E.K. 2016. Identification of small molecule binding pocket for inhibition of Crimean-Congo hemorrhagic fever virus OTU protease. Turkish Journal of Biology, 40:239-249.

%\section{\sc Conference Presentations}
%
% Kocabas, F., Ergin, E.K. 2016. Identification of small molecule binding pocket for inhibition of Crimean-Congo hemorrhagic fever virus OTU protease. Turkish Journal of Biology, 40:239-249.



\section{\sc Projects}

{\bf Anomaly Detection on CO2 levels in ISS} \hfill {\bf April 2018}\\
\vspace{-.3cm}
\begin{list2}
\item Created a tool to analyze CO2 level anomalies and clustering inside International Space Station. Presented Poster and the tool in Wearable Workshop at NASA.
\item \textbf{Tools:} Python, JavaScript, Pandas, Numpy, Scikit, Flask, HTML/CSS, Adobe Illustrator
\end{list2}


{\bf MLRPro - Machine Learning Resume Processor} \hfill {\bf April 2017}\\
\vspace{-.3cm}
\begin{list2}
\item Created a tool that evaluates the submitted resume, according to data that used to feed the machine learning algorithm behind the scene, and gives result that which level of companies given resume is qualified to apply. Worked with a team of 3.
\item \textbf{Tools:} Python, Flask, HTML/CSS, Adobe Illustrator
\item \textbf{Repository:} \href{https://github.com/MichaelMMeskhi/MLRP}{https://github.com/MichaelMMeskhi/MLRP}
\end{list2}

{\bf Open Source ACM-ICPC Preparation Curriculum} \hfill {\bf October-December 2015}\\
\vspace{-.3cm}
\begin{list2}
\item Created a curriculum to help preparation process of the international competition ACM-ICPC, to self study algorithms and data-structures and to get the underlying concepts of algorithms.
\item The curriculum is currently ranked in top 20 most popular open source courses on Github (\href{https://education.github.community/t/20-of-the-most-popular-courses-on-github/27832}{Article Link})
\item \textbf{Tools:} C++, Java, C, Python, Git, Github
\item \textbf{Repository:} \href{https://github.com/NAU-ACM/ACM-ICPC-Preparation}{https://github.com/NAU-ACM/ACM-ICPC-Preparation}
\end{list2}

{\bf Scholar Development Center} \hfill {\bf February  2016}\\
Led, organized, and implemented a project that helped students achieve increased awareness for academic success, encouraged
career readiness, and improved career opportunities through mentor program.


\section{\sc Honors and Awards}

North American University: Exceptional Merit Scholarship, 2014-2019

\vspace*{-2.5mm}
North American University: President's Honor Roll , 2016-2018


\section{\sc Extracurricular Activities}
\begin{list2}
\item ACM NAU Chapter {\em \textbf{Chair, Vice Chair, Lab Leader, Senator}} \hfill {\bf 2015-2018}
\item Artificial Intelligence, ACM-ICPC, iOS Development, {\em \textbf{Member in ACM Labs}} \hfill {\bf 2015-2018}
\item NAU Communications Club, NAU Future Leaders Club, {\em \textbf{Graphic Designer}} \hfill  {\bf 2015-2018}
\item HackNAU - 2017 | 60+ Attendees hosted, {\em \textbf{Organizer \& Director}} \hfill {\bf 2017}
\item HackHouston 2017 Best Project Overall and Best Machine Learning Project, {\em \textbf{1$^{st}$ place}} \hfill  {\bf 2017}
\item iHackathon | 30+ Attendees hosted, {\em \textbf{Organizer \& Director}} \hfill {\bf 2016} 
\item NAU - Moonlight CTF 1, {\em \textbf{Organizer \& Co-Director}} \hfill {\bf 2016}
\item ACM-ICPC Regional Contest, {\em \textbf{18$^{th}$ place}} \hfill  {\bf 2016}
\end{list2}

\section{\sc Technical Skills} 
\begin{list2}
\item{\textbf{Expert:}\space Python, C++, PyTorch, Numpy, UNIX, GitHub, Adobe Illustrator}
\item{\textbf{Advanced:}\space C, R, Linux, Git, TensorFlow, Pandas, Scikit-Learn, Java, \LaTeX, SQL, Swift, Flask, Django, HTML/CSS, Adobe Photoshop}
\end{list2}

\section{\sc References} 
{\bf References available upon request}
% \begin{list2}
% \item Associate Prof. Dr. Ricardo Vilalta, Computer Science Department, University of Houston, Houston, TX, USA, \href{rvilalta@uh.edu}{rvilalta@uh.edu}
% \item Associate Prof. Dr. Kemal Aydin, Computer Science Department, Franklin University, Columbus, Ohio, USA, \href{kemal.aydin@franklin.edu}{kemal.aydin@franklin.edu}
% \item Prof. Dr. Serkan Dursun, Researcher at Marathon Oil, Houston, TX, USA \href{sdursun@hotmail.com}{sdursun@hotmail.com} 
% \end{list2}


\end{resume}
\end{document}
